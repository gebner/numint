\documentclass{scrartcl}

%packages
\usepackage[latin9]{inputenc}
\usepackage[ngerman]{babel}
\usepackage[fleqn]{amsmath}
\usepackage{amssymb}
\usepackage{graphicx}
\usepackage{parskip}

\usepackage{listings}
\lstset{language=Octave, basicstyle=\footnotesize\tt, tabsize=8,
  breaklines=true, caption=\texttt\lstname, captionpos=b}
\DeclareFontShape{OT1}{cmtt}{bx}{n}
{<5><6><7><8><9><10><10.95><12><14.4><17.28><20.74><24.88>cmttb10}{}

\begin{document}

\section{Alternative Implementierung}

Dieser Abschnitt behandelt eine grobe, alternative Implementierung der
numerischen Integration von Funktionen $f: \mathbb R^n \to \mathbb R$
f"ur $n < \omega$ "uber einen Hyperquader $\prod_{i < n} [a_i, b_i]
\subseteq \mathbb R^n$, $a_i, b_i \in \mathbb R$.
. Die Idee ist den Integrationsbereich als Teilmenge des $\mathbb R^n$
bei Bedarf in kleinere Hyperquader zu teilen, anstatt sofort den Satz von Fubini
anzuwenden und bei der Integration der eindimensionalen Funktionen jeweils
bei Bedarf $\mathbb R$ zu teilen, in der Hoffnung so problematische
Bereiche besser isolieren zu k"onnen.

Das inh"arente Problem der Integration im Mehrdimensionalen des
exponentiell mit der Dimension wachsenden Aufwands konnte nat"urlich auch
mit dieser Implementierung nicht beseitigt werden.

Die Integration erfolgt folgender folgenderma"sen:
unterscheidet sich der relative Abstand der Werte der Integration mittels eines
Hyperw"urfelgitters mit $6^n$ und
$8^n$ St"utzstellen, dh. $6$ bzw. $8$ St"utzstellen in jeder Dimension,
wobei entlang jeder Dimension nach Newton-Cotes gewichtet aufsummiert wird
um mehr als $12 - t$ Stellen, wobei $t$ die Anzahl der bisher erfolgten
Teilungen des Integrationsbereichs ist, unterscheidet wird erneut geteilt,
ansonsten der Wert der Integration mit $8^n$ Stellen zur"uckgegeben.
Die Genauigkeitsanforderung nimmt also mit der Anzahl der Teilungen ab,
um eine Termination in vertretbarer Zeit auch bei unangenehmen Funktionen
zu erzwingen.

Die Schranke $12 - t$ wurde experimentell als im Allgemeinen vern"uftig
verifiziert, kann aber nat"urlich f"ur spezielle Klassen von Funktionen
geeignet verbessert werden. Genauso hat sich gezeigt, dass die Rechenzeit
bei Funktionen deren Ordnung den Exaktheitsgrad der Newton-Cotes-Integration
"ubersteigen, verglichen mit solchen die es nicht tun, stark zunimmt.
Ist also zu erwarten, dass vorwiegend Funktionen hoher Ordnungen
integriert werden sollen, ist eine geeignete Erh"ohung der Werte $6$ und
$8$ anzuraten.

Es sei weiters bemerkt, dass die Integration integrierbarer Funktionen
mit Singularit"aten scheitert.

\lstinputlisting{integriereQuader.m}

Die folgende Funktion ist lediglich eine Verpackung der Funktion
\lstinline|integriereQuader.m|, sodass diese nicht jedesmal explizit
mit $0$ Teilungen gestartet werden muss.

\lstinputlisting{integriereVerpackung.m}

\lstinputlisting{werteAusUndSummiere.m}

\end{document}
