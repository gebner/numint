\documentclass{scrartcl}

\usepackage[latin9]{inputenc}
\usepackage[ngerman]{babel}
\usepackage{amsmath}
\usepackage{amssymb}
\usepackage{graphicx}
\usepackage{parskip}

\usepackage{color}
\definecolor{gray}{gray}{0.50}
\usepackage{listings}
\lstset{language=Octave, basicstyle=\small, tabsize=8,frame = single,commentstyle = \color{gray},
  breaklines=true, caption=\texttt\lstname, captionpos=b}
\DeclareFontShape{OT1}{cmtt}{bx}{n}
{<5><6><7><8><9><10><10.95><12><14.4><17.28><20.74><24.88>cmttb10}{}

\pagestyle{headings}

\begin{document}

\begin{titlepage}
\title{4. Projekt f�r Numerische Mathematik UE \\
Numerische Integration �ber mehrdimensionale Gebiete}
\author{Thomas Baumhauer, Judith Braunsteiner,\\
Gabriel Ebner, Johannes Hafner,\\
Clemens M�llner, Christina Satzinger}
\maketitle
\end{titlepage}

\section{Fehlerabsch�tzungen bei mehrdimensionaler Integration}

Um eine Fehlerabsch�tzung f�r die mehrdimensionale Quadratur zu erhalten, geht man
eben so wie bei dem Verfahren mit Hilfe von Fubini vor.

Hat man nun eine $d$ dimensionale Funktion, mit eindimensionalen Quadraturen, die jeweils $n_d$ St�tzstellen haben, so
kann man folgendes beobachten:

\begin{eqnarray*}
\int_{\times [a_k,b_k]} f(x_1,...,x_d) d(x_1,...,x_d)=\int_{a_1}^{b^1} ... \int_{a_d}^{b^d} f(x_1,...,x_d) dx_d...dx_1\\
=\int_{a_1}^{b^1} ... \int_{a_{d-1}}^{b^{d-1}} \sum_{i=1}^{n_d} f(x_1,...,x_{d-1},x^i_d)g_d^i + o(d)  dx_1...dx_{d-1}\\
\end{eqnarray*}

Wobei $o(k)$ den Fehlerterm  $\int_{a_k}^{b^k} f(x_1,...,x_d) dx_k - \sum_{i=1}^{n_k} f(x_1,...,,x^i_k,...x_k)g_k^i$ bezeichnet,
also den der eindimensionalen Quadratur.

\begin{eqnarray*}
=\int_{a_1}^{b^1} ... \int_{a_{d-1}}^{b^{d-1}} \sum_{j=1}^{n_{d-1}} (\sum_{i=1}^{n_d} f(x_1,..., x_{d-2},x^j_{d-1},x^i_d)g_d^i +
o(k))g_{d-1}^i + o(d-1)  dx_1...dx_{d-2}\\
=...=\sum_{i_1=1}^{n_1}...\sum_{i_d=1}^{n_d}f(x_1^{i_1},...,x^{i_d}_d)\prod_{k=1}^d g_k^{i_k} + \sum_{k=1}^d o(k)\\
\end{eqnarray*}

Die Gewichte einer Quadratur summieren sich auf eins, wodurch sie bei den Fehlertermen wegfallen
(vorrausgesetzt nat�rlich $g_k^i\le 1 \forall i, k$).
Die erste Summe ist genau das, was das mehrdimensionale Quadraturverfahren berechnet. Die zweite Summe verbleibt als neuer Fehler.

Man sieht also, dass sich die Fehler im schlimmsten Fall aufsummieren. W�hlt man f�r alle Dimensionen das selbe Verfahren selber Ordnung,
so ergibt sich also ein Fehler der Ordnung $d \cdot o$ wobei $o$ den Fehler der eindimensionalen Quadratur bezeichnet, der im Allgemeinen von
der Intervalll�nge, der Anzahl der St�tzstellen, und h�heren Ableitungen der Funktion abh�ngt. 

\section{Programmaufbau}

Nach dem Satz von Fubini gilt f�r Integrationsgebiete $I = \times_{i = 1}^n [a_i,b_i]$:

\begin{align*}
\int_{[a_0, b_0]\times I} f(x_0, x_1, \dots, x_n) d(x_0, x_1, \dots, x_n) = 
\int_{a_0}^{b_0} \underbrace{\int_I f(x_0, x_1, \dots, x_n) d(x_1, \dots, x_n)}_{\overline{(f)(x_0)}} dx_0
\end{align*}

So l�sst sich die Integration eines mehrdimensionalen Integrals mithilfe einer Funktion \textit{dimDown} schrittweise auf die Auswertung von Integralen niedrigerer Dimension zur�ckf�hren.

\begin{align*}
f: \mathbb{R}^{n+1} \rightarrow \mathbb{R}
\end{align*}
\begin{align*}
dimDown(f):& \mathbb{R} \rightarrow \mathbb{R}\\
           & x \mapsto \int_I f(x, x_1, \dots, x_n) d(x_1, \dots, x_n)
\end{align*}

Somit reicht es grunds�tzlich skalare Funktionen integrieren zu k�nnen.\\

Der Fehler ist bei der Integration von $dimDown(f)$ durch Integration der Fehlerabsch�tzungen bei den einzelnen Integralen an den Auswertungsstellen berechenbar.

Dies wird in unserer Implementierung im �u�ersten Integrator miterledigt, wobei gleiche Ordnung, Gewichte und Auswertestellen f�r Funktion und Fehler verwendet werden.\\

Um bei Verwendung unserer optimierten Funktion $NumInt$ als Integrator auch die Statistiken f�r die Wahl des besten Verfahrens mitf�hren und aufaddieren zu k�nnen, wird unsere Funktion zu Beginn des Aufrufes um ein zus�tzliches Array $A$ erweitert.
Im skalaren Fall wird dieses einfach als $0$-Array mitgef�hrt.

\subsection{Grundstruktur}

Um im folgenden die erw�hnte Erweiterung um das Statistikarray durchzuf�hren verwenden wir im folgenden die Funktion $mapWithStats$.

\lstinputlisting{mapWithStats.m}

Unsere allgemeinere Funktion $mulDimInt$ arbeitet nun mit einem allgemeinen eindimensionalen Integrator $Integrator$ (wie etwa $NumInt$) und erwartet als Argumente au�erdem eine (mehrdimensionale) Funktion $f$, Arrays $a$ und $b$ mit unteren und oberen Integralgrenzen und eine Genauigkeitsvorgabe $epsilon$.

\lstinputlisting{mulDimInt.m}

\subsection{Integratoren}

�hnlich wir im eindimensionalen Fall haben wir die Verfahren $newton\_cotes$ und $gauss$ implementiert, die auf weitere unver�nderten Funktionen aus dem 3.Projekt zur�ckgreifen.

Es wurden hier �nderungen durchgef�hrt um die Fehler und Statistiken hochziehen zu k�nnen.

\lstinputlisting{newton_cotes.m}

\lstinputlisting{gauss.m}

Die Funktion $NumInt$ ist wiederum unser bevorzugter Integrator, der nun aus Effizienzgr�nden kein Romberg-Verfahren mehr verwendet.

\lstinputlisting{NumInt.m}

\subsection{Optimierung}

Speziell f�r den Integrator $NumInt$ haben wir die Effizienz der Berechnung von $mulDimInt$ in der Funktion $mulDimIntOpt$ durch die Vorberechnung der Arrays f�r Newton-Cotes und Gauss-Integration gesteigert.

\lstinputlisting{mulDimIntOpt.m}

\section{Beispiele}

\subsection{Laufzeit versus Dimension}

Hier haben wir eine lineare Funktion (die Summe der Koordinaten) auf dem
Einheitshyperquader integriert.

Bei der mehrdimensionalen Gauss-Quadratur beobachten wir, wie erwartet, einen
exponentiellen Laufzeitansteig, etwa mit Faktor \(10\):

\begin{verbatim}
octave:24> for i = 1:5, tic; mulDimIntGauss(@(x) sum(x),
  zeros(i,1), ones(i,1), 1e-4); toc; end
Elapsed time is 0.01991 seconds.
Elapsed time is 0.0233 seconds.
Elapsed time is 0.1175 seconds.
Elapsed time is 1.05 seconds.
Elapsed time is 10.15 seconds.
\end{verbatim}

Mit dem adaptiven Integrator beobachten wir einen deutlichen gr��eren Faktor:

\begin{verbatim}
octave:25> for i = 1:2, tic; mulDimIntOpt(@(x) sum(x),
  zeros(i,1), ones(i,1), 1e-4); toc; end
Elapsed time is 0.08288 seconds.
Elapsed time is 1.976 seconds.
\end{verbatim}

\subsection{Laufzeit versus Toleranz}

Hier sehen wir bei der Gauss-Quadratur einen moderaten Anstieg der Laufzeit mit
kleiner werdender Toleranz:

\begin{verbatim}
octave:17> for i = 4:15, tic; mulDimIntGauss(@(x) exp(x(1)*x(2)*x(3)),
  [0 0 0], [1 1 1], 10^-i); toc; end
Elapsed time is 0.1376 seconds.
Elapsed time is 0.1187 seconds.
Elapsed time is 0.1181 seconds.
Elapsed time is 0.119 seconds.
Elapsed time is 0.1184 seconds.
Elapsed time is 0.1177 seconds.
Elapsed time is 0.121 seconds.
Elapsed time is 0.1525 seconds.
Elapsed time is 0.2748 seconds.
Elapsed time is 0.2912 seconds.
Elapsed time is 0.3164 seconds.
Elapsed time is 0.3292 seconds.
\end{verbatim}

Beim adaptiven Integrator beobachten wir einen schnelleren Anstieg:

\begin{verbatim}
octave:19> for i = 4:9, tic; mulDimIntOpt(@(x) exp(x(1)*x(2)*x(3)),
  [0 0 0], [1 1 1], 10^-i); toc; end
Elapsed time is 0.9982 seconds.
Elapsed time is 1.398 seconds.
Elapsed time is 2.942 seconds.
Elapsed time is 3.654 seconds.
Elapsed time is 4.221 seconds.
Elapsed time is 7.038 seconds.
\end{verbatim}

\newpage

\section{Alternative Implementierung}

Dieser Abschnitt behandelt eine grobe, alternative Implementierung der
numerischen Integration von Funktionen $f: \mathbb R^n \to \mathbb R$
f"ur $n < \omega$ "uber einen Hyperquader $\prod_{i < n} [a_i, b_i]
\subseteq \mathbb R^n$, $a_i, b_i \in \mathbb R$.
. Die Idee ist den Integrationsbereich als Teilmenge des $\mathbb R^n$
bei Bedarf in kleinere Hyperquader zu teilen, anstatt sofort den Satz von Fubini
anzuwenden und bei der Integration der eindimensionalen Funktionen jeweils
bei Bedarf $\mathbb R$ zu teilen, in der Hoffnung so problematische
Bereiche besser isolieren zu k"onnen.

Das inh"arente Problem der Integration im Mehrdimensionalen des
exponentiell mit der Dimension wachsenden Aufwands konnte nat"urlich auch
mit dieser Implementierung nicht beseitigt werden.

Die Integration erfolgt folgender folgenderma"sen:
unterscheidet sich der relative Abstand der Werte der Integration mittels eines
Hyperw"urfelgitters mit $6^n$ und
$8^n$ St"utzstellen, dh. $6$ bzw. $8$ St"utzstellen in jeder Dimension,
wobei entlang jeder Dimension nach Newton-Cotes gewichtet aufsummiert wird
um mehr als $12 - t$ Stellen, wobei $t$ die Anzahl der bisher erfolgten
Teilungen des Integrationsbereichs ist, unterscheidet wird erneut geteilt,
ansonsten der Wert der Integration mit $8^n$ Stellen zur"uckgegeben.
Die Genauigkeitsanforderung nimmt also mit der Anzahl der Teilungen ab,
um eine Termination in vertretbarer Zeit auch bei unangenehmen Funktionen
zu erzwingen.

Die Schranke $12 - t$ wurde experimentell als im Allgemeinen vern"uftig
verifiziert, kann aber nat"urlich f"ur spezielle Klassen von Funktionen
geeignet verbessert werden. Genauso hat sich gezeigt, dass die Rechenzeit
bei Funktionen deren Ordnung den Exaktheitsgrad der Newton-Cotes-Integration
"ubersteigen, verglichen mit solchen die es nicht tun, stark zunimmt.
Ist also zu erwarten, dass vorwiegend Funktionen hoher Ordnungen
integriert werden sollen, ist eine geeignete Erh"ohung der Werte $6$ und
$8$ anzuraten.

Es sei weiters bemerkt, dass die Integration integrierbarer Funktionen
mit Singularit"aten scheitert.

\lstinputlisting{alternative_implementation/integriereQuader.m}

Die folgende Funktion ist lediglich eine Verpackung der Funktion
\lstinline|integriereQuader.m|, sodass diese nicht jedesmal explizit
mit $0$ Teilungen gestartet werden muss.

\lstinputlisting{alternative_implementation/integriereVerpackung.m}

\lstinputlisting{alternative_implementation/werteAusUndSummiere.m}


\end{document}
