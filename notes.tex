\documentclass{scrbook}
\usepackage[utf8]{inputenc}
\usepackage[ngerman]{babel}
\usepackage[fleqn]{amsmath}
\usepackage{amssymb}
\usepackage{parskip}
\usepackage{graphicx}

\usepackage{listings}
\lstset{language=Octave, basicstyle=\tt, tabsize=8,
  breaklines=true, caption=\texttt\lstname, captionpos=b}
\DeclareFontShape{OT1}{cmtt}{bx}{n}
{<5><6><7><8><9><10><10.95><12><14.4><17.28><20.74><24.88>cmttb10}{}

\begin{document}

\title{Numerische Mathematik UE -- 3. Projekt}
\author{Thomas Baumhauer, Judith Braunsteiner, Gabriel Ebner, \\
  Johannes Hafner, Clemens Müllner, Christina Satzinger}
\maketitle

\chapter{Verfahrensfehler}

\section{Newton-Côtes-Integration}

Um den Verfahrensfehler für die Interpolationsformel von Grad \(n\) zu
bestimmen, betrachten wir zunächst eine beliebige Funktion \(f \in
C^{n+1}[a,b]\), und das eindeutige Interpolationspolynom \(p\) von Grad \(n\)
an den Stützstellen \(a + (b-a)\frac{0}{n+1}, a + (b-a)\frac{1}{n+1}, \dots, a
+ (b-a)\frac{n+1}{n+1}\).  Nach Satz 4.2 aus der Vorlesung ist \(|p(x) - f(x)|
\leq \frac{\|f^{(n+1)}\|_\infty}{(n+1)!}(b-a)^{n+1}\).  Folglich gilt für die
summierte Quadraturformel auf \(C^{n+1}[a,b]\) die Abschätzung \(|Q^{\Delta}_n
f - Q f| \leq \frac {b-a} h \frac{\|f^{(n+1)}\|_\infty}{(n+1)!}h^{n+1} = (b-a)
\frac{\|f^{(n+1)}\|_\infty}{(n+1)!}h^n = O(h^n)\), wobei \(Q_n\) auf Intervalle
der Form \([a + kh, a + (k+1)h]\) angewendet wird.

% FIXME: Hermite-Bedingung verwenden

\end{document}
